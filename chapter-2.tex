\documentclass[11pt,twocolumn]{report}
\usepackage{amssymb}
\usepackage{amsmath}
\usepackage{bm} % boldface math symbols
\def\realnumbers{\mathbb{R}}
\def\naturalnumbers{\mathbb{N}}

\begin{document}
\stepcounter{chapter}
\chapter{Linear Algebra}

\textbf{Linear Algebra}
\begin{itemize}
  \item a branch of mathematics that is widely used throughout science and
    engineering
  \item a form of continuous, not discrete math
  \item read \textit{The Matrix Cookbook} (Petersen and Pedersen, 2006) or
    \textit{Linear Algebra} (Shilov, 1977) for a more thorough discussion
\end{itemize}

\section{Scalars, Vectors, Matrices, and Tensors}

\large\textbf{Scalars:}
\begin{itemize}
  \item a single number that is (by the book's notation) written in lowercase
    alpha italics
  \item "Let $ s \in \realnumbers $ be the slope of the line" while defining a
    real-valued scalar
  \item "Let $ n \in \naturalnumbers $ be the number of units" while defining a
    natural number scalar
\end{itemize}

\large\textbf{Vectors:}
\begin{itemize}
  \item an array of numbers arranged in a particular order
  \item think of vectors as things that describe the position points in space,
    with each element giving the coordinate along a particular axis
  \item one-indexed, not zero-indexed (according, again, to the book's notation)
  \item denoted as $\bm{x}$, lowercase boldface
  \item elements of $\bm{x}$ are denoted by writing its name in italic with a
    subscript, as in \textbf{$x_1, x_2$} and so on
  \item to explicitly identify the elements of a vector, write as:
    \begin{gather}
      \bm{x} = 
      \begin{bmatrix}
        x_1\\
        x_2\\
        \vdots\\
        x_n
      \end{bmatrix}
    \end{gather}
  \item we can define a set containing the indices and write the set as a
    subscript of the vector 
  \item to access $x_1, x_3, x_6$; we define set $S = \{1, 3, 6\}$ and write
    $\bm{x}_S$.
  \item we use the - sign to indicate the complement, so $x_{-S}$ is the vector
    containing all of the elements of $\bm{x}$ except for $x_1, x_2, x_3$
\end{itemize}

\large\textbf{Matrices:}
\begin{itemize}
  \item a 2-D array of numbers
  \item each element is identified by two indices in row, column order
  \item denoted in uppercase boldface alpha chars as in $\bm{A}$
  \item elements are denoted in italic but not boldface as in
    \textit{A}$_{1,1}$ is upper left entry of $\bm{A}$
  \item ":" as an index is a placeholder for "all"
  \item $\bm{A}_{i,:}$ denotes the i-th row of $\bm{A}$
  \item likewise, $\bm{A}_{:,i}$ denotes the i-th colmn of $\bm{A}$
  \item to explicitly identify the elements of a matrix, write as:
    \begin{gather}
      \bm{x} = 
      \begin{bmatrix}
        A_{1,1} & A_{1, 2}\\
        A_{2,1} & A_{2, 2}\\
      \end{bmatrix}
    \end{gather}
  \item additional notation: $f(\bm{A})_{i, j}$ gives element $(i, j)$ of the
    matrix compted by applying function $f$ to $\bm{A}$
\end{itemize}

\large\textbf{Tensors:}
\begin{itemize}
  \item in some cases we will need an array with more than two axes
  \item an array of numbers on a rectangular grid with a variable number of axes
  \item the element of tensor $\bm{A}$ at coordinates $(i, j, k)$ is
    $\bm{A}_{i, j, k}$
\end{itemize}

\section{Matrix Operations}

\end{document}
