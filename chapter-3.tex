\documentclass[11pt, twocolumn]{report}
\usepackage{amssymb}
\usepackage{amsmath}
\usepackage[margin=0.75in]{geometry}

\begin{document}
\setcounter{chapter}{2}
\chapter{Probability and Information Theory}
Probability Theory
\begin{itemize}
  \item a mathematical framework for representing uncertain statements
  \item provides a means of quantifing uncertainty
  \item allows us to reason in the presence of uncertainty
  \item this is useful for AI systems because the laws of probability tell us
    how they should reason, also for analyzing their behavior
  \item was originally developed to analyze the frequencies of events
  \item defines a set of formal rules for determining the likelihood of a
    proposition being true given the likelihood of other propositions
\end{itemize}
Information Theory
\begin{itemize}
  \item allows us to quantify the amount of uncertainty in a probability
    distribution
\end{itemize}

\subsection{Why Probability?}
\begin{itemize}
  \item even though many branches of computer science deal mostly with entities
    that are deterministic/certain, machine learning ALWAYS deals with
    uncertain and sometimes stochastic quantities
  \item nearly all activities require some ability to reason amidst uncertainty
  \item in fact it is hard to think of examples beyond mathematical statements
    (that are true by definition) 
\end{itemize}

\textbf{3} possible sources of uncertainty:
\begin{enumerate}
  \item inherent stochasticity in the system being modeled
  \item incomplete observability (even deterministic systems can appear
    stochastic when we cannot observe all the variables the drive the system's
    behavior)
  \item incomplete modeling (when we use a model that must discard some of the
    information we have observed, the discarded information can result in
    uncertainty in the model's predictions)
\end{enumerate}

\textbf{Frequentist probability}:
\begin{itemize}
  \item related directly to the rates at which \textit{repeatable} events occur
  \item events like drawing a certain hand of cards in a poker game (repeatable)
\end{itemize}

\textbf{Bayesian probability}:
\begin{itemize}
  \item related to the qualitative levels of uncertainty (\textbf{degree of
      belief})
  \item like when a doctor says that a patient has a 40\% chance of having the
    flu (degree of belief = 40 percent)
\end{itemize}

It is proven that Bayesian probabilities are controlled by the same axioms as
the ones that govern Frequentist probabilities. (Ramsey, 1926)
\end{document}
