\documentclass[11pt, twocolumn]{report}
\usepackage[margin=0.75in]{geometry}
\usepackage{amssymb}
\usepackage{amsmath}
\usepackage{bm}
\def\realnumbers{\mathbb{R}}

\begin{document}
\setcounter{chapter}{4}
\chapter{Machine Learning Basics}

Overview:
\begin{itemize}
  \item what is a learning algorithm?
  \item fitting training data (to a model) is a different beast altogether from
    finding new patterns that generalize to new data
  \item for more reading, read Murphy (2012) or Bishop (2006)
  \item most ML algorithms have settings called \textit{hyperparameters}
  \item \textbf{machine learning is essentially a form of applied statistics
      with an increased emphasis on the use of computers to statistically
      estimate complicated functions and a decreased emphasis on proving
      confidence intervals around these functions}
  \item frequentist estimators and Bayseian inference
  \item supervised vs unsupervised learning
  \item most ML algorithms are based on an optimization algorithm called
    \textbf{stochastic gradient descent}
  \item the main components for an ML algorithm are: an optimization algorithm,
    a cost function, a model, a dataset
  \item factors that limit the ability of traditional machine learning to
    generalize
\end{itemize}

\section{Learning Algorithms}
A machine learning algorithm is an algorithm that is able to learn from data.
But what do we mean by \textit{learning}? Mitchell (1997) provides a succinct
definition:

"A computer program is said to learn from experience $E$ with
respect to some class of tasks $T$ and performance measure $P$, if its
performance at tasks in $T$ (as measured by $P$) improves with experience $E$."

\subsection{The Task, $T$}
\begin{itemize}
  \item the process of learning itself is \textbf{not} the task
  \item learning is the means of attaining the ability to perform the task
  \item if we want a robot to walk, then walking is the task
  \item machine learning tasks are usually describe in terms of how the machine
    learning system should process an \textbf{example}
  \item an example is a collection of \textbf{features} that have been
    quantitatively measured from some object or event that we want the machine
    learning to process
  \item we typically represent an example as a vector $\bm{x} \in
    \realnumbers^n$ where each entry $x_i$ of the vector is another feature
  \item the features of an image are usually the value of the pixels in the
    image
  \item there are many kinds of tasks that can be solved with machine
    learning
\end{itemize}

\subsubsection{Examples of tasks}

Classification
\begin{itemize}
  \item specify which of $k$ categories some input belongs to
  \item we ask the system to produce some function $f : \realnumbers^n \to
    \{1,...,k\}$
  \item when $y = f(\bm{x})$, assigns an input described by vector $\bm{x}$ to
    a category identified by numeric code $y$ to a category identified by
    numeric code $y$
  \item an example is an input image, and the output is a numeric code
    identifying the object in the image
  \item object recognition is the same basic tech as facial recognition
\end{itemize}

Regression
\begin{itemize}
  \item predict a numerical value given some input
  \item $f : \realnumbers^n \to \realnumbers$
  \item similar to classification save for the output (categorical vs numeric)
  \item an example is the prediction of future prices of produce
  \item these kinds of predictions are used for algorithmic trading
\end{itemize}

Transcription
\begin{itemize}
  \item 
\end{itemize}


\end{document}

